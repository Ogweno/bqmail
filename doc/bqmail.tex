\documentclass[12pt, a4paper]{report}

%%%%%%%%%% Load Package %%%%%%%%%%%
\usepackage[top=3.0cm,bottom=2.0cm,left=3.5cm,right=3.5cm]{geometry}
\usepackage{fontspec}
\usepackage[noblocks,affil-it]{authblk}
\usepackage[colorlinks = true,
	    linkcolor = cyan,
            citecolor = blue,
            anchorcolor = blue]{hyperref}
\usepackage{listings}
\usepackage{booktabs}
\usepackage{titlesec}
\usepackage{geometry}
%%%%%%%%%% Set Format %%%%%%%%%%%%%%
\setmainfont{Times New Roman}
\newfontfamily{\C}{Consolas}
\newfontfamily{\tb}{Times New Roman Bold}
\newfontfamily{\ti}{Times New Roman Italic}
\lstset{basicstyle=\C\footnotesize,breaklines=true,tabsize=4}
\linespread{1.3}
\titleformat{\chapter}{\Huge\bfseries}{\thechapter}{1em}{}
%%%%%%%%%% First Page %%%%%%%%%%%%%%
\title{BQMail\\User Manual\\Version 2.3.2}
\author[1,*]{Mijian Xu}
\affil[1]{\small School of Earth Science and Engineering, Nanjing University}
\affil[*]{\small Email: gomijianxu@gmail.com}

%%%%%%%%%%% Doc Begin %%%%%%%%%%%%%%%
\begin{document}
\maketitle
\tableofcontents

%===================================%
%------------ Chapter 1 ------------%
%===================================%
\chapter{Introduction}

BQMail is an open source software package for requesting seismic data from Incorporated Research Institutions for Seismology (IRIS) Data Management Center (DMC) with the BREQ\_FAST service (\url{http://ds.iris.edu/ds/nodes/dmc/manuals/breq_fast/}). BREQ\_FAST is a popular method for accessing to the IRIS DMC archive via electronically mailings. Users first set parameters (e.g., station name, date range, file format) in command lines, and then create formatted files and send to the IRIS DMC automatically with the BQMail package. Meanwhile, users can use it for searching stations with necessary parameters in command lines.

Scripts in the BQMail package were developed with Python programming language (\url{https://www.python.org}) on OSX 10.10 platform. They are compatible with both Python 2.7 and Python 3.x. The BQMail package runs on OSX and Linux/Unix platform, but is not tested under Windows. It is distributed under the GNU General Public License Version 3 (GPLv3) as published by the Free Software Foundation (\url{http://www.gnu.org/licenses/gpl.html}).

%===================================%
%------------ Chapter 2 ------------%
%===================================%
\chapter{Installation}

%----------- Section 2.1 -----------%
\section{Dependencies}
BQMail depends on standard libraries of Python 2.7 or higher versions, which include datetime, os, re, smtplib, urllib, sys, getopt, glob, ConfigParser/configParser and math.

%----------- Section 2.2 -----------%
\section{Installation}
\subsection{Download BQMail}
After opening a terminal, run the following commands:
\begin{lstlisting}
	git clone git://github.com/xumi1993/bqmail.git
\end{lstlisting}
\subsection{Install BQMail}
In the root directory of BQMail, users may use the package by running the scripts. To use it in any other directory, just run:
\begin{lstlisting}
	cd bqmail
	./install.sh
\end{lstlisting}

%===================================%
%------------ Chapter 3 ------------%
%===================================%
\chapter{Tutorial}
%----------- Section 3.1 -----------%
\section{bqmail}
bqmail - Request seismic waveform data.
\subsection{Synopsis}
{\tb bqmail} {\tb -N}{\ti network} {\tb -S}{\ti station} {\tb -Y}{\ti ymin/mmin/dmin/ymax/mmax/dmax} {\tb -B}{\ti sec\_before/sec\_after} [{\tb -C}{\ti channel}] [{\tb -L}{\ti location}] [{\tb -c}{\ti datetimefile}] [{\tb -F[{\ti seed}|{\ti miniseed}]}] {\ti configfile}
\subsection{Required Arguments}
\begin{description}
\item[{\ti configfile}] Specify directory of config file, which contains a events list, BREQ\_FAST options, and information of electronic mail server. The table \ref{tab31} lists options in the config file.
\begin{table}
\caption{Options in config file}
\centering
\begin{tabular}{c|c}\toprule[1.5 pt]
Option & Function \\ \midrule[1 pt]
{\C eventlst} & Directory of the formatted events list.\\
{\C NAME} &  Folder name at IRIS DMC ftp site.\\
{\C INST} & Institution.\\
{\C EMAIL} & Email address to send and receive related mail.\\
{\C MEDIA} & Media for accessing data. [Default is {\C Electronic (FTP)}.]\\
{\C hosts} & Host name of SMTP server.\\
{\C port} & Port of the SMTP server. [Default is {\C 25}.]\\
{\C passwd} & Clear text password of the {\C EMAIL}.\\
\bottomrule[1.5 pt]
\end{tabular}
\label{tab31}
\end{table}
\item[{\tb -N}{\ti network}] Specify network code.
\item[{\tb -S}{\ti station}] Select a station of specified network by {\tb -N}
\item[{\tb -Y}{\ti ymin/mmin/dmin/ymax/mmax/dmax}] Select a date range during the archive time.
\item[{\tb -B}{\ti sec\_before/sec\_after}] Set time duration of each seismogram from {\ti sec\_before} before to {\ti sec\_after} after event time in seconds
\end{description}
\subsection{Optional Arguments}
\begin{description}
\item[{\tb -C}{\ti channel}] Specify channels like "{\ti ?H?}" or "{\ti HHZ}". [Default is "{\ti BH?}"]. 
\item[{\tb -L}{\ti location}] Location identifier
\item[{\tb -c}{\ti datetimefile}] If this argument is specified, {\tb -Y} will be futile. The time range will be specified in a table file with 12 column as: [{\ti year1 month1 day1 hour1 minute1 sec1 year2 month2 day2 hour2 minute2 sec2}].
\item[{\tb -F[{\ti seed}|{\ti miniseed}]}] Select a format (seed or miniseed) to retrieve [Default is {\ti seed}].
\end{description}
\subsection{Example}
To request waveform data by events of CB.NJ2 station from 2013 to 2014, try:
\begin{lstlisting}
	bqmail -NCB -SNJ2 -Y2013/1/1/2014/12/31 -B0/3600 head.cfg
\end{lstlisting} 

%----------- Section 3.2 -----------%
\section{bqmail\_conti}
bqmail\_conti - Request continuous seismic waveform data by hours.
\subsection{Synopsis}
{\tb bqmail\_conti} {\tb -I}{\ti stationlist} {\tb -Y}{\ti ymin/mmin/dmin/ymax/mmax/dmax} {\tb -H}{\ti hours} [{\tb -C}{\ti channel}] [{\tb -F[{\ti seed}|{\ti miniseed}]}] {\ti configfile}
\subsection{Required Arguments}
\begin{description}
\item[{\ti configfile}] Specify directory of config file. The config file contains a events list, BREQ\_FAST options, and information of electronic mail server. Table \ref{tab31} lists options in the config file.
\item[{\tb -I}{\ti stationlist}] Select a text file including network \& station information as:[{\ti network station [location]}]
\item[{\tb -Y}{\ti ymin/mmin/dmin/ymax/mmax/dmax}] Select a date range during the archive time.
\item[{\tb -H}{\ti hours}] Specify a time duration of each retrieving data file in hours.
\end{description}
\subsection{Optional Arguments}
\begin{description}
\item[{\tb -C}{\ti channel}] Specify channels like "{\ti ?H?}" or "{\ti HHZ}". [Default is "{\ti BH?}"]. 
\item[{\tb -F[{\ti seed}|{\ti miniseed}]}] Select a format (seed or miniseed) of retrieving data file. [Default is {\ti seed}].
\end{description}
\subsection{Example}
To request continuous seismic waveform data with format of miniseed from 1 Jan. 2015 to 1 Jan. 2015 every 1 day, try:
\begin{lstlisting}
	bqmail_conti -Ista.lst -Y2015/1/1/2015/1/5 -H24 -Fminiseed \
	head.cfg
\end{lstlisting}
this is a record of the {\ti sta.lst}
\begin{lstlisting}
	CB NJ2
	CB TNC
	IC BJT 00
\end{lstlisting}

%----------- Section 3.3 -----------%
\section{searchDMC}
searchDMC - Find stations in IRIS DMC. Stations defined by different criterions. First, using {\tb -R} to find stations in a box region. Second, using {\tb -D} to find stations in a specified region by epicentral distance. Third, using {\tb -N [-S]} to find stations under a specified network.
\subsection{Synopsis}
{\tb searchDMC} [{\tb -N}{\ti network}] [{\tb -S}{\ti station}] [{\tb -R}{\ti lonmin/lonmax/latmin/latmax}] \\\
[{\tb -D}{\ti lon/lat/dismin/dismax}] [{\tb -Y}{\ti ymin/mmin/dmin/ymax/mmax/dmax}] [{\tb -C}{\ti channel}] [{\tb -K}]
\subsection{Arguments}
\begin{description}
\item[{\tb -N}{\ti network}] Specify a network code.
\item[{\tb -S}{\ti station}] Select a station under the specified network by {\tb -N}
\item[{\tb -R}{\ti lonmin/lonmax/latmin/latmax}] Limits stations in a box region. Latitude is from -90$^\circ$ to 90$^\circ$  and longitude is from -180$^\circ$ to 180$^\circ$.
\item[{\tb -D}{\ti lon/lat/dismin/dismax}] Limits station in a specified region by epicentral distance between {\ti dismin} and {\ti dismax} from a center at {\ti lat, lon}. The distance is from 0 to 180 degrees.
\item[{\tb -Y}{\ti ymin/mmin/dmin/ymax/mmax/dmax}] Select a date range during the archive time.
\item[{\tb -C}{\ti channel}] Specify a channel like "{\ti BHZ}". This argument with unsupported wildcard is different from that in {\tb bqmail} (or {\tb bqmail\_conti}).
\item[{\tb -K}] Generates a KML file in current directory. which is used by Google Earth to display stations and related information based on IRIS DMC metadata. The argument of {\tb -D} does not support this function.
\end{description}
\subsection{Example}
To find stations in a box region from 2002 to 2004, use
\begin{lstlisting}
	searchDMC -R90/100/20/30 -Y2002/1/1/2004/12/31
\end{lstlisting}
To find stations in the region with epicentral distance between 0$^\circ$ and 10$^\circ$ from a center at 25$^\circ$N and 100$^\circ$E, use
\begin{lstlisting}
	searchDMC -D100/25/0/10
\end{lstlisting}
To find stations under network {\C IC} with channel of {\C HHZ} and Generate a KML file, use
\begin{lstlisting}
	searchDMC -NIC -CHHZ -K
\end{lstlisting}

%----------- Section 3.4 -----------%
\section{updateCatalog}
updateCatalog - Automatically update the events list from Harvard CMT Catalog.
\subsection{Synopsis}
{\tb ubdateCatalog} {\tb -I}{\ti inputfile} [{\tb -O}{\ti outputfile}]
\subsection{Required Arguments}
\begin{description}
\item[{\tb -I}{\ti inputfile}] Specify the directory of events list.
\end{description}
\subsection{Optional Arguments}
\begin{description}
\item[{\tb -O}{\ti outputfile}] Specify a directory of the updated events list. If it is not specified, the {\ti inputfile} will be overwritten as a updated events list.
\end{description}
\subsection{Example}
To update the the events list, use
\begin{lstlisting}
	updateCatalog -I~/work/EventCMT.dat -O/tmp/Newlist.dat 
\end{lstlisting}
\end{document}







	
